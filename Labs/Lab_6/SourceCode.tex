\documentclass{article}
\usepackage[T2A]{fontenc}			% кодировка
\usepackage[utf8]{inputenc}			% кодировка исходного текста
\usepackage[english,russian]{babel}

\usepackage[dvipsnames]{xcolor}

\usepackage{graphicx}
\usepackage{float}
\usepackage{wrapfig}



\title{Техническое задание на создание автоматизированной системы «Документооборот»}
\author{Николай Сухоруков, Салыгина Ирина }
\date{ Декабрь 2018}

\begin{document}

\maketitle

\title{Разделы технического задания}

\begin{enumerate}
    \item Общие сведения
    \item Назначение и цели создания системы
    \item Характеристика объектов автоматизации
    \item Требования к системе
    \item Состав и содержание работ по созданию системы
    \item Структура и сценарии использовании приложения
    \item Список источников
\end{enumerate}

\newpage

\section{Общие сведения}
    \subsection{Наименование системы}
        \subsubsection{Полное наименование системы}
            \paragraph{Полное наименование: Корпоративная система документооборота}
        \subsubsection{Краткое наименование системы}
            \paragraph{Краткое наименование системы: КСД, Система}
    
    \subsection{Основания для проведения работ}
        \paragraph{Работа выполняется на основании лабораторной работы №6}
    
    \subsection{Наименование организаций – Заказчика и Разработчика}
        \subsubsection{Заказчик}
            \paragraph{Заказчик: ФГБОУ ВО «ВолгГТУ»}
            \paragraph {Адрес фактический: г. Волгоград, пр. Ленина, 28 }
       
        \subsubsection{Разработчик}
            \paragraph{Разработчик: ФГБОУ ВО «ВолгГТУ», кафедра САПРиПК}
            \paragraph {Адрес фактический: г. Волгоград, пр. Ленина, 28 }
            
    \subsection{Плановые сроки начала и окончания работы}
        \paragraph{Начало работы: 12 января 2019 г.}
        \paragraph {Окончание работы: 1 апреля 2019 г. }
        
    \subsection{Источники и порядок финансирования}
        \paragraph{Работа выполняется на безвозмездной основе}
        
    \subsection{Порядок оформления и предъявления заказчику результатов работ}
        \paragraph{Работы по созданию Системы сдаются Разработчиком поэтапно в соответствии с календарным планом проекта. По окончании каждого их этапов работ Разработчик сдает Заказчику соответствующие отчетные документы этапа.}

\section{Назначение и цели создания системы}
    \subsection{Назначение системы}
        \paragraph{Система предназначена для повышения оперативности и качества создания и обработки документации Заказчика. Основным назначением Системы является автоматизация основных бизнес-процессов, связанных с обработкой и хранением информации Заказчика.В рамках проекта автоматизируется информационная деятельность в следующих бизнес-процессах: а) Ведение табеля; б) Подготовка отчета }
    \subsection{Цели создания системы}
        \paragraph{Система создается с целью}
        \begin{itemize}
            \item обеспечения сбора и первичной обработки исходной информации, необходимой для подготовки отчетности;
            \item создания единой системы отчетности;
            \item повышения качества информации.
        \end{itemize}
        
        \paragraph{В результате создания и внедрения Системы должны быть улучшены значения следующих показателей:}
        \begin{itemize}
            \item время сбора и первичной обработки исходной информации;
            \item количество информационных систем, используемых для подготовки отчетности;
            \item время, затрачиваемое на информационную деятельность.
        \end{itemize}
   
\section{Характеристика объектов автоматизации}
    \paragraph{Логика работы Системы подразумевает работу с четырьмя подразделениями}

\begin{table}[h!]
  \caption{Подразделения организации}
  \begin{tabular}{c|c|c}
    \textbf{Подразделение} & \textbf{Что делает?} & \textbf{Возможна ли автоматизация?} \\ % Column rows are separated by &
    \hline % a horizontal line
     Проектный офис & Контроль за документооборотом & Частично \\
    \hline
        Библиотека & Выдача списка статей & Возможна \\
    \hline
        Кафедра САПРиПК & Оформляет документы & Частично \\
    \hline
        Бухгалтерия & Выдача табелей & Возможна \\
  \end{tabular}
\end{table}
    
    
\section{Требования к системе}
    \subsection{Требования к системе в целом}
        \subsubsection{Требования к структуре и функционированию системы}
            \paragraph{Система должна быть централизованной, т.е. все данные должны располагаться в центральном хранилище. Система должна иметь трехуровневую архитектуру (первый – источник, второй – хранилище, третий – источник). В системе планируется выделить следующие функциональные подсистемы:}
            
        \begin{itemize}
            \item подсистема сбора, обработки и загрузки данных, предназначенная для реализации процессов сбора данных из систем источников, приведения указанных данных к виду, необходимому для наполнения подсистемы хранения данных;
            \item подсистема хранения данных, которая предназначена для хранения данных;
            \item подсистема формирования и визуализации отчетности, которая предназначена для накопления и визуализации данных.
        \end{itemize}
        
        \paragraph{В качестве протокола взаимодействия между компонентами Системы на транспортно-сетевом уровне необходимо использовать протокол TCP/IP.}
        \paragraph{Для организации информационного обмена между компонентами Системы должны использоваться специальные протокол прикладного уровня, например, HTTP или HTTPS. }
        
        \paragraph{Смежными системами для Системы являются информационные системы планирования. Источниками данных для Системы должны быть:}
        
        \begin{itemize}
            \item Информационная система управления предприятием (СУБД MS SQL);
            \item Информационная система обеспечения бюджетного процесса (СУБД Oracle).
        \end{itemize}    
        
        \paragraph{Система должна поддерживать следующие режимы функционирования: }
        \begin{itemize}
            \item Основной режим, в котором подсистемы КСД выполняют все свои основные функции;
            \item Профилактический режим, в котором одна или все подсистемы КХД не выполняют своих функций;
        \end{itemize} 
        
        \paragraph{В основном режиме функционирования Система КХД должна обеспечивать:}
        \begin{itemize}
            \item работу пользователей режиме – 24 часов в день, 7 дней в неделю (24х7);
            \item выполнение своих функций – сбор, обработка и загрузка данных; хранение данных, предоставление отчетности.
        \end{itemize}
        
        \paragraph{В профилактическом режиме Система КХД должна обеспечивать возможность проведения следующих работ:}
        \begin{itemize}
            \item техническое обслуживание;
            \item модернизацию аппаратно-программного комплекса;
            \item устранение аварийных ситуаций.
        \end{itemize}
    
    \subsubsection{Требования к численности и квалификации персонала системы и режиму его работы}
        \paragraph{В состав персонала, необходимого для обеспечения эксплуатации Системы в рамках соответствующих подразделений Заказчика, необходимо выделение следующих ответственных лиц:}
        \begin{itemize}
            \item Руководитель эксплуатирующего подразделения - 1 человек;
            \item Администратор подсистемы сбора, обработки и загрузки данных - 2 человека;
            \item Администратор подсистемы хранения данных - 2 человека;
            \item Администратор подсистемы формирования и визуализации отчетности - 1 человек.
        \end{itemize}
        
        \subsubsection{Требования к квалификации персонала}
            \paragraph{К квалификации персонала, эксплуатирующего Систему, предъявляются следующие требования.}
            
             \begin{itemize}
                \item Конечный пользователь - знание соответствующей предметной области; знание основ многомерного анализа; знания и навыки работы с аналитическими приложениями;
                \item Администратор подсистемы сбора, обработки и загрузки данных – знание методологии проектирования хранилищ данных; знание интерфейсов интеграции ХД с источниками данных; знание СУБД; знание языка запросов SQL;
                \item Администратор подсистемы хранения данных - глубокие знания СУБД; знание архитектур; опыт администрирования СУБД; знание и навыки операций архивирования и восстановления данных; знание и навыки оптимизации работы СУБД;
                \item Администратор подсистемы формирования и визуализации отчетности – понимание принципов многомерного анализа; знание методологии проектирования хранилищ данных; знание и навыки администрирования приложения; знание языка запросов SQL; знание инструментов разработки.
            \end{itemize}
        
        \subsubsection{Требования к надежности}
            \paragraph{Уровень надежности должен достигаться согласованным применением организационных, организационно-технических мероприятий и программно-аппаратных средств. Надежность должна обеспечиваться за счет:}
            \begin{itemize}
                \item применения технических средств, системного и базового программного обеспечения, соответствующих классу решаемых задач;
                \item своевременного выполнения процессов администрирования Системы КХД;
                \item соблюдения правил эксплуатации и технического обслуживания программно-аппаратных средств;
                \item предварительного обучения пользователей и обслуживающего персонала.
            \end{itemize}
            
            \paragraph{Обеспечение информационное безопасности Системы КХД должно удовлетворять следующим требованиям:}
            \begin{itemize}
                \item Защита Системы должна обеспечиваться комплексом программно-технических средств и поддерживающих их организационных мер;
                \item Защита Системы должна обеспечиваться на всех технологических этапах обработки информации и во всех режимах функционирования, в том числе при проведении ремонтных и регламентных работ;
                \item Программно-технические средства защиты не должны существенно ухудшать основные функциональные характеристики Системы (надежность, быстродействие, возможность изменения конфигурации);
                \item Разграничение прав доступа пользователей и администраторов Системы должно строиться по принципу "что не разрешено, то запрещено";
            \end{itemize}
        
    \newpage
        
    \subsection{Требования к видам обеспечения}
        \subsubsection{Требования к программному обеспечению}
 
            \paragraph{К обеспечению качества ПС предъявляются следующие требования:}
            \begin{itemize}
                \item функциональность должна обеспечиваться выполнением подсистемами всех их функций;
                \item надежность должна обеспечиваться за счет предупреждения ошибок - не допущения ошибок в готовых ПС;
                \item легкость применения должна обеспечиваться за счет применения покупных программных средств;
                \item эффективность должна обеспечиваться за счет принятия подходящих, верных решений на разных этапах разработки ПС и системы в целом;
                \item сопровождаемость должна обеспечиваться за счет высокого качества документации по сопровождению, а также за счет использования в программном тексте описания объектов и комментариев; использованием осмысленных и устойчиво различимых имен объектов; размещением не больше одного оператора в строке текста программы; 
            \end{itemize}
            
        \subsubsection{Требования к техническому обеспечению}
            \paragraph{Система должна быть реализована с использованием специально выделенных серверов Заказчика.}
            \paragraph{Сервер базы данных должен быть развернут на HP9000, минимальная конфигурация которого должна быть: CPU: 16 (32 core); RAM: 128 Gb; HDD: 500 Gb; Network Card: 2 (2 Gbit); Fiber Channel: 4.}
            
\section{Состав и содержание работ по созданию системы}
    \paragraph{Работы по созданию системы выполняются в три этапа:}
    \begin{itemize}
        \item Проектирование. Разработка эскизного проекта. Разработка технического проекта (продолжительность — X месяца);
        \item Разработка рабочей документации. Адаптация программ (продолжительность — Y месяцев);
        \item Ввод в действие (продолжительность — Z месяца).
    \end{itemize}
    
\section{Структура и сценарии использовании приложения}
    \subsection{Архитектура системы}
        \paragraph{Система состоит из подсистем обработки данных, генерации отчетов, базы данных и внешнего источника данных.}
    
        \begin{figure}[ht]
            \centering
            \includegraphics[width=0.8\linewidth]{arch.png}
            \caption{Архитектура системы}
            \label{fig:mpr}
        \end{figure}
    
    \subsection{Основные сценарии использования приложения}
        \paragraph{После авторизации пользователю системы будут доступны следующие функции:}
        \begin{itemize}
            \item выход из системы;
            \item запрос в библиотеку на получение списка статей;
            \item онлайн редактирование шаблонов документа;
            \item импорт документа в файловое хранилище;
            \item импорт данных в систему 1С;
            \item предпросмотр и загрузка документов.
        \end{itemize}
        
        \begin{figure}[ht]
            \centering
            \includegraphics[width=0.8\linewidth]{usecase.png}
            \caption{Use-Case диаграмма использования}
            \label{fig:usecase}
        \end{figure}
        
    \subsection{Структура приложения}
        \subsubsection{Главная страница}
            \paragraph{Главная страница представляет собой страницу входа в систему}
            
            \begin{figure}[ht]
                \centering
                \includegraphics[width=0.8\linewidth]{login.png}
                \caption{Страница входа в систему}
                \label{fig:login}
            \end{figure}
  
   \newpage
   
        \subsubsection{Управление документооборотом}
            \paragraph{Страница управления документооборотом предлагает пользователю выполнить в определенной последовательности ряд действий по формированию отчета.}
            
            \begin{figure}[ht]
                \centering
                \includegraphics[width=0.8\linewidth]{conseq.png}
                \caption{Работа с системой}
                \label{fig:howtouse}
            \end{figure}
            
\newpage


\cite{5768414}.
\cite{7480156}.
\cite{5768412}
\cite{6568293}
\cite{5768459}

\bibliographystyle{plain}
\bibliography{mybib}



\end{document}
